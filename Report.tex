\documentclass{article}
\usepackage[utf8]{inputenc}

\title{CS455 Assignment 3}
\author{Anand Singh Kunwar (13110) }
\date{October 2016}

\usepackage{natbib}
\usepackage{graphicx}


% Default fixed font does not support bold face
\DeclareFixedFont{\ttb}{T1}{txtt}{bx}{n}{12} % for bold
\DeclareFixedFont{\ttm}{T1}{txtt}{m}{n}{12}  % for normal

% Custom colors
\usepackage{color}
\definecolor{deepblue}{rgb}{0,0,0.5}
\definecolor{deepred}{rgb}{0.6,0,0}
\definecolor{deepgreen}{rgb}{0,0.5,0}

\usepackage{listings}

% Python style for highlighting
\newcommand\pythonstyle{\lstset{
language=Python,
basicstyle=\ttm,
otherkeywords={self},             % Add keywords here
keywordstyle=\ttb\color{deepblue},
emph={MyClass,__init__},          % Custom highlighting
emphstyle=\ttb\color{deepred},    % Custom highlighting style
stringstyle=\color{deepgreen},
frame=tb,                         % Any extra options here
showstringspaces=false            % 

}}


% Python environment
\lstnewenvironment{python}[1][]
{
\pythonstyle
\lstset{#1}

}
{}

% Python for external files
\newcommand\pythonexternal[2][]{{
\pythonstyle
\lstinputlisting[#1]{#2}
}}

% Python for inline
\newcommand\pythoninline[1]{{\pythonstyle\lstinline!#1!}}

\begin{document}

\maketitle

\section{Use-Case/Purpose}
This code is a very crude example of a graphic rendering engine. However for the
simplicity of the problem and demonstration we aren't dealing with graphics and
we render text-animation. Here in the example we render the famous ROFLCopter
animation as a series of Frames. We have 2 files $IoC.py$ and $Traditional.py$
each corresponding to its title.
\begin{python}[caption=IoC.py]
import time                                                                     
class Frame(object):                                                            
    def __init__(self, Frame):                                                  
        self.Frame = Frame                                                      
                                                                                
class Engine(object):                                                           
    def __init__(self):                                                         
        # Default Frame Rate is 24 fps                                          
        self.FrameRate = 24                                                     
        self.Scene = []                                                         
        self.Loop = True                                                        

    def AddFrame(self, frame):                                                  
        """For Adding a Frame to the Scene"""                                   
        self.Scene.append(frame)                                                

    def RenderFrame(self, frame):                                               
        """A Crude Frame Renderer"""                                            
        print frame.Frame                                                       

    def Render(self):                                                           
        """Render Function to be called by main"""                              
        while True:                                                             
            for frame in self.Scene:                                            
                time.sleep(1.0/self.FrameRate)                                  
                print '\n'*50   # Clearing Frame Hack                           
                self.RenderFrame(frame)                                         
            if not self.Loop:                                                   
                break                                                           
def main():                                                                     
    engine = Engine()                                                           

    ROFLCopter1 = open('roflcopter1', 'r').read()                               
    ROFLCopter2 = open('roflcopter2', 'r').read()                               

    engine.AddFrame(Frame(ROFLCopter1))                                         
    engine.AddFrame(Frame(ROFLCopter2))                                         
    engine.Render()                                                             

if __name__ == '__main__':                                                      
    main() 
\end{python}

\begin{python}[caption=Tradition.py]
import time                                                                     
class Frame(object):                                                            
    def __init__(self, Frame):                                                  
        self.Frame = Frame                                                      
def main():                                                                     
    ROFLCopter1 = open('roflcopter1', 'r').read()                               
    ROFLCopter2 = open('roflcopter2', 'r').read()                               
    Scene = []                                                                  
    FrameRate = 24                                                              
    Loop = True                                                                 
                                                                                
    # Constructing Scene                                                        
    Scene.append(Frame(ROFLCopter1))                                            
    Scene.append(Frame(ROFLCopter2))                                            
                                                                                
    # Rendering Here                                                            
    while True:                                                                 
        for frame in Scene:                                                     
            time.sleep(1.0/FrameRate)                                           
            print '\n'*50   # Clearing Frame Hack                               
            print frame.Frame                                                   
        # Loop Condition Check                                                  
        if not Loop:                                                            
            break                                                               
                                                                                
if __name__ == '__main__':                                                      
    main()   
\end{python}

\section{Advantages of Inversion of Control over Traditional}
Although the code may seem more in our application in the case of Inversion of
Control, this adds quality attributes like testability, modifiability etc. We
can have default configurations in our Engine Part. If we already have the
Engine/Framework ready we can make our application (here the ROFLCopter
animation) with just lesser lines of code in comparison to our Traditional Way.
The control is with the Engine/Framework. The drawing/rendering part is all
handled by the engine and when to call what. We only supply the
configurations/scenes and the rest of the jobs are handled by the framework
itself. There is even an option of default configuration built into the engine
which can be altered if wanted. Hence this provides an easy to make application
by using the Engine without much configuration and code (Look at the main
block). 

\end{document}

